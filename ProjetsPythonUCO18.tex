%\RequirePackage{atbegshi}
\documentclass[french]{beamer}
\usepackage{etex}

\usepackage[beamer,utf8x]{preambuleTrm}



\usepackage{iwona}


\mode<presentation>
{
  %\usetheme{Bergen}%Montpellier,Madrid}
  % or ...Frankfurt, 
  \usetheme[secheader]{Montpellier}

  %\setbeamercovered{highly dynamic}
  % or whatever (possibly just delete it)
  \usefonttheme{progressbar}
%  \useoutertheme{progressbar}
  %\useinnertheme{progressbar}
 % \progressbaroptions{% titlepage=normal,
  %  imagename=/home/moi/Photos/Maths/prisoner
%}
}

%\usepackage{elephantbird}
\newcommand\MyBox[2][XXX]{%
  \fbox{\tabular[b]{C{#1}} #2 \endtabular}}

\setbeamertemplate{footline}[frame number]
\beamertemplatenavigationsymbolsempty

%\usepackage[french,vlined,boxed]{algorithm2e}
\usepackage{bookmark,multido}
%\usepackage{xlop}

\ifpdf \DeclareGraphicsRule{*}{mps}{*}{} \fi
\usepackage{tikz, pgfkeys, pgfopts, xstring}
\usetikzlibrary{automata,fit,trees,matrix,arrows,decorations.markings,shapes,arrows,chains,positioning,intersections,backgrounds,calc,through,mindmap,shadows,patterns, positioning}

\usepackage{tikz-uml}

\usepackage[underline=true,rounded corners=false]{pgf-umlsd}

\hypersetup{pdfpagemode=FullScreen}

\usepackage[detect-all]{siunitx}

\usepackage{caption}
\captionsetup{labelformat=empty,font=footnotesize}

\usepackage[square]{daedale}
\bibliographystyle{cyclope}


\usepackage{algo}

\renewcommand{\algocommentfont}{\tiny\ttfamily\itshape}
\renewcommand{\algokeywordfont}{\small\ttfamily}
\renewcommand{\algotextfont}{\footnotesize\ttfamily}
\newtheorem{exercice}{Exercice}


\setlength{\columnseprule}{0pt}

%\setlength{\parskip}{0pt}


\graphicspath{{/home/moi/IUT/Figures/}{/home/moi/VISA_III/figures_visa/}{/home/moi/Lycee/TDmaple/2006_7/}{/home/moi/Figures/Arbres_Graphes/}{/home/moi/Figures/FigSTI/}{/home/moi/Figures/FigMaple/}{/home/moi/Photos/Maths/}{/home/moi/Photos/Tehessin/}{/home/moi/Figures/FigSTI/}{/home/moi/Figures/FigSeconde/}{/home/moi/Lycee/Informatique/XCAS/2008_9/}{/home/moi/Lycee/TDmaple/2008_9/}{/home/moi/IUT/Thierry/Conversions/}{/home/moi/Photos/informathix/}{/home/moi/Lycee/Informatique/PafAlgo/}{/home/moi/Figures/figuresAlice/}{/home/moi/Figures/FigSeconde/}{/home/moi/Biocam/robot/}{/home/moi/Biocam/CoursIni3d/}{/home/moi/Biocam/central/figures/}{/home/moi/Biocam/central/}{/home/moi/IUT/INFO2/Images/}{/home/moi/IUT/INFO2/2011_12/graphes/}{/home/moi/IUT/INFO1/CM1/}{/home/moi/Figures/figuresTS/}{/home/moi/Photos/Tehessin/}{/home/moi/Figures/FigSTI/}{/home/moi/Alice/figures3/}{/home/moi/Alice/figures3/newcourbes/}{/home/moi/Alice/figures/}{/home/moi/Figures/figuresAlice/}{/home/moi/Photos/Maths/}{/home/moi/Lycee/TS/}{/home/moi/Lycee/TS/2009_10/}{/home/moi/Figures/FigMaple/}{/home/moi/Figures/figuresTS/newcourbes/}{/home/moi/Figures/FigEspace/}{/home/moi/PROG/HASKELL/FLOAT/}{/home/moi/LYCEE/LIL/4e/2016_17/Images/}{/home/moi/LYCEE/LIL/2nde/2016_17/Images/}{/home/moi/LYCEE/LIL/2nde/2016_17/fonctions/}{/home/moi/LYCEE/LIL/1S/2016_17/Images/}{/home/moi/LYCEE/LIL/1S/2016_17/Images/}{/home/moi/LYCEE/LIL/1S/2016_17/suites/suites1es/}{/home/moi/LYCEE/LIL/1S/2016_17/trigo/}{/home/moi/Photos/Maths/CAPTURES_TI/Suites/}{/home/moi/LYCEE/FIGURES/}{/home/moi/UCO/L1/ANGERS2018/Images/}}



\setcounter{tocdepth}{2} %\setcounter{page}{0}




\usepackage{fdsymbol}





\setbeamercolor{block title}{use=structure,fg=black,bg=blue!55!white}
\setbeamercolor{block body}{use=structure,fg=black,bg=blue!10!white}


\setbeamercolor{block title example}{use=structure,fg=black,bg=green!75!white}
\setbeamercolor{block body example}{use=structure,fg=black,bg=green!10!white}

\setbeamercolor{block title alert}{use=structure,fg=black,bg=red!75!white}
\setbeamercolor{block body alert}{use=structure,fg=black,bg=red!10!white}



\setbeamerfont{block body}{size*={10}{12}}
\setbeamerfont{block body alerted}{size*={10}{12}}
\setbeamerfont{block body example}{size*={10}{12}}


\renewenvironment{exercise}{\begin{exampleblock}{Recherche}}{\end{exampleblock}}

\usepackage{scalefnt}
\setbeamertemplate{theorems}[normal size]


\begin{document}


\title[]% (optional, use only with long paper titles)
{Compléments d'informatique}

\subtitle{L1 MIASHS}

\author[] % (optional, use only with lots of authors)
{Guillaume CONNAN }
% - Give the names in the same order as the appear in the paper.
% - Use the inst{?} command only if the authors have different
%   affiliation.

\institute{Université Catholique de l'Ouest - Angers}% (optional, but mostly needed)

\logo{\includegraphics[width=0.1\textwidth]{logoUCO}}

%\logo{\includegraphics[scale=0.15]{big_connan}}

\date[] % (optional, should be abbreviation of conference name)
{Novembre-Décembre 2018}
% - Either use conference name or its abbreviation.
% - Not really informative to the audience, more for people (including
%   yourself) who are reading the slides online

\subject{ }



\beamerdefaultoverlayspecification{<+->}


\AtBeginSubsection[]
{\scriptsize
  \begin{frame}<beamer>
    \frametitle{Sommaire}
 {\tiny
\begin{multicols}{2}
    \tableofcontents[currentsection,currentsubsection]
       \end{multicols}
}

  \end{frame}
}





\AtBeginSection[]
{
  \begin{frame}<beamer>
    \frametitle{Sommaire}
 {\scriptsize
\begin{multicols}{2}
    \tableofcontents[currentsection]
       \end{multicols}
}

  \end{frame}
}


% If you wish to uncover everything in a step-wise fashion, uncomment
% the following command: 

\beamerdefaultoverlayspecification{<+->}




\newcommand{\TR}{\mathcal{T}}


\begin{frame}
%\ThisCenterWallPaper{1}{logoiremBigInv}
 \Large \titlepage 
\end{frame}

\begin{frame}
 \frametitle{Sommaire}
{\scriptsize
\begin{multicols}{2} 
 \tableofcontents
\end{multicols}
}

 
 \end{frame}

\normalsize


% début



\section{Descriptif}

\begin{frame}
  \begin{enumerate}
  \item Modélisation d'une application
  \item Découverte et utilisation des objets
  \item Documentation d'une application
  \item Compléments d'algorithmique
  \item Réalisation d'un projet en binôme
  \end{enumerate}
\end{frame}


\section{Modélisation}

\begin{frame}
  \begin{quote}
    Avant donc que d'écrire, apprenez à penser.\\
Selon que notre idée est plus ou moins obscure,\\
L'expression la suit, ou moins nette, ou plus pure.\\
Ce que l'on conçoit bien s'énonce clairement.\\
...\\
Hâtez-vous lentement, et, sans perdre courage,\\
Vingt fois sur le métier remettez votre ouvrage\\
Polissez-le sans cesse et le repolissez ;\\
Ajoutez quelquefois, et souvent effacez. \\
\end{quote}

\pause

\begin{flushright}
  Nicolas \textsc{Boileau} in \textit{L'Art poétique} - 1674
\end{flushright}
\end{frame}

\begin{frame}
  
%\begin{figure}
\begin{center}
  \includegraphics[width=\linewidth]{numerobis}
\end{center}
%\end{figure}

\end{frame}


\begin{frame}\frametitle{Un cas simple d'étude}
Le service d'état-civil de la mairie  d'Angers demande un outil pour enregistrer
des  personnes  et pour  mettre  à  jour  des  changements dans  leur  situation
(mariage, décés, etc.)

\pause

Nous allons  nous restreindre dans un  premier temps au cas  de l'enregistrement
d'une personne et d'un éventuel mariage.
\end{frame}




\begin{frame}
  \begin{itemize}
  \item UML  est le langage  de modélisation orienté objet  le plus connu  et le
    plus utilisé;
  \item Objectif: cadre unifié dans un contexte industriel;
    \item Succès: simplicité ! structures intuitives;
    \item UML 2.0 est apparu en 2002.
     
  \end{itemize}
\end{frame}

\begin{frame}
  UML : un problème $\longrightarrow$ un \textbf{diagramme}
  \begin{itemize}
  \item diagramme de classe
  \item diagramme de cas d'utilisation
  \item diagramme de déploiement
  \item diagramme de séquence
  \item diagramme d'activité
    \item ... une vingtaine en tout!
  \end{itemize}
\end{frame}


\begin{frame}
  
%\begin{figure}
\begin{center}
  \includegraphics[width=\linewidth]{DiagramXKCD}
\end{center}
%\end{figure}

\end{frame}


\begin{frame}
  \begin{enumerate}
  \item Réfléchir
  \item Définir la structure \og gros grain\fg{}
  \item Documenter
  \item Guider le développement
  \item Tester
  \end{enumerate}
\end{frame}


\begin{frame}
  MAIS
 \begin{itemize}
  \item  les logiciels  sont  de plus  en  plus complexes  et  ont besoin  d'être
    sécurisés;
  \item il faut de la rigueur et de la précision;
    \item techniques fiables à base mathématique;
  \item \textbf{Méthodes formelles}: méthode B, Esterel, Coq,...
  \end{itemize}
\end{frame}


\begin{frame}
  
%\begin{figure}
\begin{center}
  \includegraphics[width=\linewidth]{TabFormellesUml}
\end{center}
%\end{figure}

\end{frame}


\section{Vision statique : diagramme de Classes}


\begin{frame}
  
%\begin{figure}
\begin{center}
  \includegraphics[height=\textheight]{DiagIllisisble}
\end{center}
%\end{figure}

\end{frame}


\begin{frame}
  
 \begin{itemize}\frametitle{Classes}
  \item  Une diagramme de classes donne une vue graphique de la structure statique d'un
    système;
    \item Une  classe représente  la strucuture  commune d'un  ensemble d'objets:
      personne, vélo, roue, cadre...
      \item Une  classe est représentée par  un rectangle, son nom  commence par
        une majuscule et le rectangle peut contenir trois parties
      \end{itemize}
    \end{frame}


\begin{frame}
  \begin{center}
  \begin{tikzpicture}
    \umlemptyclass{Vélo}
  \end{tikzpicture}
\end{center}
\end{frame}


\begin{frame}\frametitle{Attributs}
  \begin{itemize}
\item Un  attribut est  une caractéristique, une  information contenue  dans une
  classe;
  \item  la taille,  la  marque,  le modèle,  l'année  de  fabrication sont  des
    attributs de vélo
    \item  Moins bricolo:  un attribut  est une  relation binaire  d'un ensemble
      d'objets vers un ensemble de valeurs:

      \pause Voiture numéro 3 \pause (\textit{Objet})\pause \ a\_pour\_couleur
      \pause (\textit{Attribut})\pause \ Jaune \pause (\textit{Valeur})
\end{itemize}
\end{frame}


\begin{frame}
  \begin{center}
  \begin{tikzpicture}
    \umlclass{Vélo}{Marque: String\\ Modèle: String\\ Année: Integer\\ /Âge: Integer}{}
  \end{tikzpicture}
\end{center}
\end{frame}

\begin{frame}\frametitle{Opérations}

  \begin{itemize}
  \item Une  opération est un service  qu'un objet (une \textit{instance}  de la
    classe) peut exécuter;
    \item fonctionner, en\_réparation
  \end{itemize}
  
\end{frame}



\begin{frame}
  \begin{center}
  \begin{tikzpicture}
    \umlclass{Personne}{Nom:  String\\ Naissance: Integer\\  /Âge:
      Integer}{se\_marier(p: Personne): void }
  \end{tikzpicture}
\end{center}
\end{frame}




\begin{frame}\frametitle{Héritage}
  \begin{itemize}
  \item c'est une relation entre une classe et une classe plus générale;
    \item un vélo est un moyen de transport individuel, la voiture aussi,le tram
      est un moyen transport  en commun, un moyen de transport  en commun est un
      moyen de transport...
  \end{itemize}
\end{frame}


\begin{frame}
  \begin{center}
    \begin{tikzpicture}
      \umlsimpleclass{Transport}
      \umlsimpleclass[x=-2, y=-3, anchor=north]{Transport individuel}
      \umlsimpleclass[x=3, y=-3, anchor=north]{Transport commun}
      \umlsimpleclass[x=3, y=-6, anchor=north]{Bus}
      \umlsimpleclass[x=-4, y=-6, anchor=north]{Vélo}
      \umlsimpleclass[x=0, y=-6, anchor=north]{Voiture}
       \umlVHVinherit{Transport commun}{Transport}
       \umlVHVinherit{Transport individuel}{Transport}
       \umlVHVinherit{Bus}{Transport commun}
       \umlVHVinherit{Voiture}{Transport individuel}
       \umlVHVinherit{Vélo}{Transport individuel}
    \end{tikzpicture}
  \end{center}
\end{frame}

\begin{frame}\frametitle{Associations}
  \begin{itemize}
  \item une association sera de  manière privilégiée binaire: elle connecte deux
    éléments (classes le plus souvent)
  \item une associations binaire a deux \textit{association ends}
  \item une association est paramétrée par au moins un des éléments suivants:
    \begin{itemize}
    \item un nom (minuscule) décrivant le rôle joué par une entité par rapport à
      l'autre. Si l'association est bidirectionnelle, un verbe à l'actif dans un
      sens est lu au passif dans l'autre sens;
    \item Une multiplicité (0,1, *, 1..*, 0..2, etc.)
    \item un genre d'agrégation dans le cas où l'association n'est pas symétrique:
      \begin{itemize}
      \item  \textit{composite}($\vardiamondsuit$):  la destruction  d'une  entité
        entraîne la destruction  de l'autre. Un élément ne  peut appartenir qu'à
        un seul agrégat \textit{composite} (chat / pattes)
       \item simple ($\diamond$) (voiture/roue)
      \end{itemize}
      
    \end{itemize}
  \end{itemize}
\end{frame}

\begin{frame}
  \begin{center}
  \begin{tikzpicture}
    \umlsimpleclass{Université}
    \umlsimpleclass[x=6]{Faculté}
    \umlsimpleclass[y=-3]{Étudiant}
    \umlsimpleclass[x=6,y=-3]{Cours}
    \umlcompo[mult1=1, mult2=1..*]{Université}{Faculté}
    \umlaggreg[mult1=1.., mult2=*]{Université}{Étudiant}
    \umlassoc[mult1=1..*,stereo=suit,mult2=1..*]{Étudiant}{Cours}
  \end{tikzpicture}
\end{center}
\end{frame}

\begin{frame}
  
%\begin{figure}
\begin{center}
  \includegraphics[width=\linewidth]{AccesBatiment}
\end{center}
%\end{figure}

\end{frame}

\begin{frame}\frametitle{Contraintes et notes}
  \begin{itemize}
  \item Pour arriver à  un semblant de sérieux par rapport  aux méthodes formelles, on
  doit compléter le diagramme de classe par des indications de contraintes.

\item Trop souvent, ces contraintes sont écrites en langage naturel.

  \item Il faut utiliser le langage OCL (\textit{Object Constraint Language}) pour éviter le bricolage.
  \end{itemize}
  
\end{frame}


\begin{frame}
  \begin{quote}
    1842 - Ada Lovelace writes the first program. She is hampered in her efforts
    by the minor inconvenience that she doesn't have any actual computers to run
    her code. Enterprise  architects will later relearn her  techniques in order
    to program in UML. 
  \end{quote}
\end{frame}


\begin{frame}
  \begin{itemize}
  \item Vous sautez de l'avion;
    \item Puis vous  lisez la contrainte: \og avant de  sauter s'assurer d'avoir
      enfilé un parachute opérationnel\fg{}
    \item Trop tard...
      \item en passant: interprété/compilé
  \end{itemize}
\end{frame}


\begin{frame}
  
%\begin{figure}
\begin{center}
  \includegraphics[width=\linewidth]{Shadock}
\end{center}
%\end{figure}

\end{frame}


\begin{frame}
  \begin{center}
    \begin{quote}
      L'administration aura  désormais deux mois  pour répondre au  courrier des
      usagers: les fonctionnaires ont choisi juin et novembre.
    \end{quote}
  \end{center}
\end{frame}


\begin{frame}
  \begin{center}
  \begin{tikzpicture}
    \umlsimpleclass{Personne}
    \umlassoc[arg=marier, angle1=0, angle2=60, mult1=0..1, loopsize=3cm]{Personne}{Personne}
  \end{tikzpicture}
\end{center}
  \pause

  \begin{center}
    Quels problèmes?
  \end{center}

\end{frame}


\begin{frame}
  \begin{center}
  \begin{tikzpicture}
    \umlsimpleclass{Personne}
    \umlassoc[arg=marier, angle1=0, angle2=60, mult1=0..1, loopsize=3cm]{Personne}{Personne}
    \umlnote[y=-2, width=7cm]{Personne}{Il faut être majeur, ne pas se marier à soi-même,
      ne pas se marier si on est déjà marié}
  \end{tikzpicture}
\end{center}
\end{frame}



\begin{frame}
  \textcolor{white}{.}
  \vspace{-2cm}
  \begin{center}
  \begin{tikzpicture}
    \umlsimpleclass[y=3]{Personne}
    \umlassoc[arg=marier, angle1=0, angle2=60, mult1=0..1, loopsize=3cm]{Personne}{Personne}
    \umlnote[y=0, width=10cm]{Personne}{{\tiny
      \texttt{context Personne::marier(p :Personne)\\
        pre pas\_marier\_soi-meme: not(p = self)\\
        - - aucune des personnes ne doit avoir été mariée\\
        pre pas\_deja\_marie : self.spouse->size() = 0 and p.spouse->size()=0\\
        pre majeur : self.age >= 18 and p.age >= 18\\
        post : self.spouse = p and p.spouse = self
      }}
    }
  \end{tikzpicture}
\end{center}
\end{frame}


\section{Python}

\begin{frame}
  \begin{quote}
    1991  -  Dutch programmer  Guido  van  Rossum  travels  to Argentina  for  a
    mysterious operation. He returns with  a large cranial scar, invents Python,
    is declared Dictator for Life by  legions of followers, and announces to the
    world that "There Is Only One Way to Do It." Poland becomes nervous. 
  \end{quote}
\end{frame}



\begin{frame}
  
%\begin{figure}
\begin{center}
  \includegraphics[height=.7\textheight]{pygravity}
\end{center}
%\end{figure}
\pause
%\begin{figure}
\begin{center}
  \includegraphics[height=.2\textheight]{pygravity2}
\end{center}
%\end{figure}

\end{frame}






\section{Juste assez de POO pour traduire notre UML en actes}


\begin{frame}
  
Vous  vous   êtes  sûrement  demandés   pourquoi  les  appels   de  \og
fonctions\fg{} n'ont pas toujours la même syntaxe...

\end{frame}



\begin{frame}[fragile]
\begin{pythoncode}
Moi[1]: xs = [1,2]

Moi[2]: append(xs, 3)
Traceback (most recent call last):

  File "<ipython-input-2-c65410c1a3c8>", line 1, in <module>
    append(xs, 3)

NameError: name 'append' is not defined
\end{pythoncode}
\end{frame}


\begin{frame}[fragile]
  \begin{pythoncode}
Moi[3]: xs.append(3)

Moi[4]: xs
Python[4]: [1, 2, 3]
\end{pythoncode}
\end{frame}





\begin{frame}[fragile]
  \begin{pythoncode}

Moi[6]: xs.len()
Traceback (most recent call last):

  File "<ipython-input-6-bedaf6ec6921>", line 1, in <module>
    xs.len()

AttributeError: 'list' object has no attribute 'len'
\end{pythoncode}
\end{frame}


\begin{frame}[fragile]
\begin{pythoncode}
Moi[7]: len(xs)
Python[7]: 3
\end{pythoncode}
  
\end{frame}



\begin{frame}[fragile]
  
Même les nombres ont ce genre de \og fonctions\fg{} bizarres qui suivent un
point:

\pause

\begin{pythoncode}
Moi[10]: 1.2.is_integer()
Python[10]: False

Moi[11]: 1.0.is_integer()
Python[11]: True
\end{pythoncode}


  
\end{frame}



\begin{frame}
  

Une \textit{classe} correspond à un \og moule\fg{} permettant de créer
un type d'objet.

Il ne faut pas confondre la classe avec l'objet (il vaut mieux manger
le gâteau que le moule).

\end{frame}


\begin{frame}
  
Prenons  un exemple:  vous  êtes Saroumane  et vous  avez  un moule  à
orques.     Vous      pouvez     créer     ainsi      une     infinité
d'orques.   Informatiquement    la   classe    \texttt{Orque}   permet
d'\textit{instancier} un objet de type orque.


\pause


Un orque a un nom et un poids: ce sont des \textit{attributs} (ses caractéristiques). Il peut
effectuer un salut: c'est une \textit{méthode} (ce que l'objet peut faire).

\end{frame}




\begin{frame}
  \begin{center}
  \begin{tikzpicture}
    \begin{umlclass}{Orque}{nom \\ poids \\ maitre}{salut()}
    \end{umlclass}
  \end{tikzpicture}
\end{center}
\end{frame}


\begin{frame}[fragile]
  \begin{pythoncode}
class Orque:

    nom    = ""
    poids  = 0
    maitre = ""

    def salut(self):
        print("Ash nazg durbatulûk! Mon nom est %s" % self.nom)
\end{pythoncode}


  
\end{frame}




\begin{frame}[fragile]
\begin{pythoncode}
In [76]: a = Orque()

In [77]: a.nom = "Grishnákh"

In [78]: a.salut()
Ash nazg durbatulûk! Mon nom est Grishnákh

In [79]: a.age = 2

In [80]: b = Orque()

In [81]: b.nom = "Uglúk"

In [82]: b.salut()
Ash nazg durbatulûk! Mon nom est Uglúk

In [83]: a.maitre = "Saroumane"

In [84]: b.maitre
Out[84]: 'Sauron'
\end{pythoncode}


\end{frame}



\begin{frame}[fragile]
  \begin{pythoncode}
class Orque:

    def __init__(self):
        self.nom    = ""
        self.poids  = 0
        self.maitre = "Sauron"

    def salut(self):
        print("Ash nazg durbatulûk! Mon nom est %s" % self.nom)
\end{pythoncode}


  
\end{frame}



\begin{frame}[fragile]
  \begin{pythoncode}
class Orque:

    def __init__(self, nom, poids, maitre = "Sauron"):
        self.nom    = nom
        self.poids  = poids
        self.maitre = maitre

    def salut(self):
        print("Ash nazg durbatulûk! Mon nom est %s" % self.nom)
\end{pythoncode}


  
\end{frame}


\begin{frame}[fragile]
  \begin{pythoncode}
In [106]: a = Orque("Truc",3)

In [107]: a.maitre
Out[107]: 'Sauron'

In [108]: a = Orque("Truc","Sauron")

In [109]: a.maitre
Out[109]: 'Sauron'

In [110]: a.poids
Out[110]: 'Sauron'

In [111]: a = Orque("Truc",100,"Sauron")

In [112]: a.poids
Out[112]: 100

In [113]: a.maitre
Out[113]: 'Sauron'

In [114]: a.nom
Out[114]: 'Truc'
\end{pythoncode}


  
\end{frame}




\begin{frame}[fragile]
  \begin{pythoncode}
class Orque:

    def __init__(self, nom:str, poids:int, maitre:str = "Sauron") -> None:
        self.nom    = nom
        self.poids  = poids
        self.maitre = maitre

    def salut(self)->None:
        print("Ash nazg durbatulûk! Mon nom est %s" % self.nom)
\end{pythoncode}


\end{frame}


  \begin{frame}[fragile]
    \begin{pythoncode}
Moi[22]: a = Orque("Truc","Sauron")

Moi[23]: a.poids
Python[23]: 'Sauron'
\end{pythoncode}
  \end{frame}





\begin{frame}[fragile]
  \begin{pythoncode}
class Orque:

    def __init__(self, nom:str, poids:int, maitre:str = "Sauron") -> None :
        assert type(poids) == int, "le poids est un entier"
        self.nom    = nom
        self.poids  = poids
        self.maitre = maitre

    def salut(self)->None:
        print("Ash nazg durbatulûk! Mon nom est %s" % self.nom)
\end{pythoncode}


\end{frame}


  \begin{frame}[fragile]
    \begin{pythoncode}
Moi[25]: a = Orque("Truc","Sauron")
Traceback (most recent call last):

  File "<ipython-input-25-4b1feddb2229>", line 1, in <module>
    a = Orque("Truc","Sauron")

  File "/home/moi/UCO/L1/ANGERS2018/Python/Orques.py", line 12, in __init__
    assert type(poids) == int, "le poids est un entier"

AssertionError: le poids est un entier
\end{pythoncode}
  \end{frame}




  \begin{frame}
    
Imaginez maintenant  que vous  êtes JRR  \textsc{Tolkien} et  que vous
voulez fabriquer  des Hobbits. Vous  vous dites qu'il suffit  de créer
une classe \texttt{Hobbit}:

\pause

\begin{center}
  \begin{tikzpicture}
    \begin{umlclass}{Hobbit}{nom \\ prenom \\ poids}{salut()}
    \end{umlclass}
  \end{tikzpicture}
\end{center}
  \end{frame}


  \begin{frame}[fragile]
    \begin{pythoncode}
class Hobbit:

    def __init__(self, nom:str, prenom:str, poids:int) -> None:
        assert type(poids) == int, "le poids est un entier"
        assert type(nom) == str and type(prenom) == str, "Le nom et le prénom sont des String"
        self.nom    = nom
        self.prenom = prenom
        self.poids  = poids

    def salut(self) -> None:
        print("Bonjour! Mon nom est %s %s" % (self.prenom, self.nom))
\end{pythoncode}


    
  \end{frame}

  \begin{frame}[fragile]
    \begin{pythoncode}
Moi[27]: a = Hobbit("Sacquet", "Bilbo", 50)

Moi[28]: a.salut()
Bonjour! Mon nom est Bilbo Sacquet
\end{pythoncode}


  \end{frame}




  \begin{frame}
    
Et ensuite il y a les Elfes, les Nains, etc.

On remarque cependant  que les Orques comme les Hobbits  ont un nom et
un poids. 

On  peut  alors créer  une  \og  super-classe\fg{} \texttt{Etre}  dont
\texttt{Hobbit}   et  \texttt{Orque}   serait  des   sous-classes  qui
\textit{héritent} de ses attributs...comme en UML.
  \end{frame}


  \begin{frame}[fragile]
    \begin{pythoncode}
class Etre(object):

    def __init__(self, appellation:str, poids:int, bonjour:str) -> None:
        assert type(poids) == int, "le poids est un entier"
        assert type(appellation) == str and type(bonjour) == str, "Le nom et la salutation sont des String"
        self.appellation = appellation
        self.poids       = poids
        self.bonjour     = bonjour

    def salut(self) -> None:
        print("%s ! Mon nom est %s" % (self.bonjour, self.appellation))

    def masse(self) -> None:
        print("Je pèse %f kg" % self.poids)


class Orque(Etre):

    def __init__(self, nom:str, poids:int, maitre = "Sauron"):
        Etre.__init__(self,nom,poids,"Ash nazg durbatulûk")
        self.maitre = maitre

class Hobbit(Etre):

    def __init__(self, nom:str, prenom:str, poids:int):
        Etre.__init__(self,prenom + ' ' + nom, poids, "Bonjour")
        self.prenom  = prenom
\end{pythoncode}


    
  \end{frame}



  \begin{frame}[fragile]
    \begin{pythoncode}
Moi[31]: a = Hobbit("Sacquet", "Bilbo", 50)

Moi[32]: a.masse()
Je pèse 50.000000 kg

Moi[33]: a.salut()
Bonjour ! Mon nom est Bilbo Sacquet

Moi[34]: b = Orque("Uglúk",100)

Moi[35]: b.salut()
Ash nazg durbatulûk ! Mon nom est Uglúk

Moi[36]: c = Orque("Uglúk","Saroumane")
Traceback (most recent call last):

  File "<ipython-input-36-44938988904b>", line 1, in <module>
    c = Orque("Uglúk","Saroumane")

  File "/home/moi/UCO/L1/ANGERS2018/Python/Orques.py", line 29, in __init__
    Etre.__init__(self,nom,poids,"Ash nazg durbatulûk")

  File "/home/moi/UCO/L1/ANGERS2018/Python/Orques.py", line 13, in __init__
    assert type(poids) == int, "le poids est un entier"

AssertionError: le poids est un entier
\end{pythoncode}


  \end{frame}



  \section{Diagramme de cas d'utilisation}


  \begin{frame}
    
%\begin{figure}
\begin{center}
  \includegraphics[width=\linewidth]{UseCase1}
\end{center}
%\end{figure}

\pause

<<include>> : je fais toujours 
\pause
<<extend>> : je choisis de faire

\pause comment savoir si je choisis de faire une action?

\end{frame}



\section{Diagramme d'activité}


\begin{frame}
  
%\begin{figure}
\begin{center}
  \includegraphics[width=\linewidth]{ReveilEtudiantActivite}
\end{center}
%\end{figure}
\end{frame}


  \section{Diagramme de séquence}


  \begin{frame}\frametitle{Diagramme de séquence}
    \begin{itemize}
    \item Représente une interaction entre des instances qui jouent des rôles.
    \item Dynamique car permet de visualiser une durée de vie.
    \item  L'interaction  se fait  par  envoi  de \og  messages\fg{}:  création,
      destruction...
    \item Sémantique un peu floue...
    \end{itemize}
   
  \end{frame}



  \begin{frame}
    \begin{sequencediagram}
\newthread{t}{Instance 1: Abonné}
\newinst[1]{s}{Le Système:}
\begin{call}{t}{ Chercher un livre() }{s}{}
\end{call}
\begin{call}{t}{ Authentification() }{s}{}
\end{call}
\begin{call}{t}{ Emprunter() }{s}{}
  \begin{callself}{s}{Vérifier les demandes()}{}
  \end{callself}
\end{call}

\end{sequencediagram}
  \end{frame}


\section{GIT}


\begin{frame}
  Vous créerez en TD un dépôt GIT pour travailler à deux sur un même projet.

  
%\begin{figure}
\begin{center}
  \includegraphics[width=\linewidth]{GitHub}
\end{center}
%\end{figure}

\end{frame}


\begin{frame}
  {\LARGE \href{https://github.com/}{https://github.com/}}

  
%\begin{figure}
\begin{center}
  \includegraphics[width=\linewidth]{GitHub2}
\end{center}
%\end{figure}

\end{frame}


\section{Dictionnaires}


\subsection{Découverte}

\begin{frame}[fragile]
  Ensemble contenant des éléments auxquels on  a accès à l'aide de clés choisies
  par le créateur...

  \pause

  \begin{pythoncode}
In [45]: Tel = {}

In [46]: Tel['Roger'] = '06 12 11 13 20'

In [47]: Tel
Out[47]: {'Roger': '06 12 11 13 20'}

In [48]: Tel['Roger']
Out[48]: '06 12 11 13 20'
\end{pythoncode}

  
\end{frame}



\begin{frame}[fragile]
\begin{pythoncode}
In [49]: Tel['Josette'] = '07 00 00 01 00'

In [50]: Tel['Bill'] = '06 05 04 03 02'

In [51]: Tel
Out[51]: 
{'Bill': '06 05 04 03 02',
 'Josette': '07 00 00 01 00',
 'Roger': '06 12 11 13 20'}

In [52]: les06 = {ami for ami in Tel if Tel[ami][:2] == '06'}

In [53]: les06
Out[53]: {'Bill', 'Roger'}
\end{pythoncode}


\end{frame}


\begin{frame}[fragile]
\begin{pythoncode}
In [54]: del Tel['Josette']

In [55]: Tel
Out[55]: {'Bill': '06 05 04 03 02', 'Roger': '06 12 11 13 20'}
\end{pythoncode}

  \pause

\begin{pythoncode}
In [56]: Tel.keys()
Out[56]: dict_keys(['Bill', 'Roger'])

In [57]: Tel.values()
Out[57]: dict_values(['06 05 04 03 02', '06 12 11 13 20'])
\end{pythoncode}


\end{frame}



\subsection{Travail à faire}


\begin{frame}
  Revenons à notre problème d'état-civil.

  \pause

  On veut créer une table qui tient à jour les personnes inscrites.

  \pause

  De quoi a-t-on besoin?

  
\end{frame}

\begin{frame}[fragile]
\begin{pythoncode}
class Table(object):

    def __init__(self) -> None:
        self.idmax = 0 # identifiant maxi courant
        self.taille = 0 # taille du dictionnaire
        self.table = {} # le dictionnaire
        self.inscrits = self.table.values() # les inscrits

    def __repr__(self) -> str:
        t = self.table
        return str({k: str(t[k]) for k in t.keys()})

    def __getitem__(self, k) -> Inscrit:
        """ 
        Pour pouvoir utiliser les crochets
        Si T est un objet Table
        T[10] renvoie l'inscrit portant l'identifiant 10
        """
        return self.table[k]


    def ajoute(self, p:Inscrit) -> None:


    def enleve(self, p:Inscrit) -> None:
   
\end{pythoncode}


\end{frame}




\section{Lecture / Écriture de fichiers}

\begin{frame}
  On dispose de tableaux de données au format CSV
\end{frame}

\begin{frame}
  
%\begin{figure}
\begin{center}
  \includegraphics[width=\linewidth]{mib}
\end{center}
%\end{figure}

\end{frame}



\begin{frame}

%\begin{figure}
\begin{center}
  \includegraphics[width=\linewidth]{TableAlien}
\end{center}
%\end{figure}

\end{frame}

\begin{frame}[fragile]

CSV : Comma Separated Values
  
\begin{minted}{text}
Nom,Sexe,Planete,NoCabine
Zorglub,M,Trantor,1
Blorx,M,Euterpe,2
Urxiz,M,Aurora,3
Zbleurdite,F,Trantor,4
Darneurane,M,Trantor,4
Mulzo,M,Helicon,6
Zzzzzz,F,Aurora,7
Arghh,M,Nexon,8
Joranum,F,Euterpe,9
 \end{minted}
\end{frame}

\begin{frame}[fragile]
\begin{pythoncode}
BaseAliens = {
        Alien('Zorglub', 'M', 'Trantor', '1'),
        Alien('Blorx', 'M', 'Euterpe', '2'),
        Alien('Urxiz', 'M', 'Aurora', '3'),
        Alien('Zbleurdite', 'F', 'Trantor', '4'),
        Alien('Darneurane', 'M', 'Trantor', '4'),
        Alien('Mulzo', 'M', 'Helicon', '6'),
        Alien('Zzzzzz', 'F', 'Aurora', '7'),
        Alien('Arghh', 'M', 'Nexon', '8'),
        Alien('Joranum', 'F', 'Euterpe', '9')
}
\end{pythoncode}
\end{frame}



\begin{frame}[fragile]
\begin{pythoncode}
class Alien:
	def __init__(self, Nom, Sexe, Planete, NoCabine):
		self.Nom = Nom
		self.Sexe = Sexe
		self.Planete = Planete
		self.NoCabine = NoCabine
\end{pythoncode}
\end{frame}


\begin{frame}[fragile]
\begin{minted}{text}
 /home/moi/UCO/L1/ANGERS2018/Python/MIB_Files:
  total used in directory 32 available 60318924
  drwxr-xr-x 2 moi moi 4096 Nov 20 20:53 .
  drwxr-xr-x 5 moi moi 4096 Nov 20 23:45 ..
  -rw-r--r-- 1 moi moi  149 Nov 20 20:45 BaseAgents.csv
  -rw-r--r-- 1 moi moi  199 Nov 20 20:46 BaseAliens.csv
  -rw-r--r-- 1 moi moi   53 Nov 20 20:46 BaseCabines.csv
  -rw-r--r-- 1 moi moi   98 Nov 20 10:53 BaseGardiens.csv
  -rw-r--r-- 1 moi moi  163 Nov 20 20:46 BaseMiams.csv
  -rw-r--r-- 1 moi moi   31 Nov 20 20:46 BaseResponsables.csv
\end{minted}

\pause
  
\begin{pythoncode}
mes_csv_file = {Path(f).stem:open(f,"r") for f in glob.glob("./MIB_Files/*.csv")}
\end{pythoncode}


\end{frame}



\begin{frame}[fragile]
\begin{pythoncode}
import glob
# The glob module finds all the pathnames matching a specified pattern according to the rules used by the Unix shell
from pathlib import Path
# The final path component, without its suffix
\end{pythoncode}


\end{frame}





\begin{frame}[fragile]
  \begin{pythoncode}
In [58]: mes_csv_file
Out[59]: 
{'BaseAgents': <_io.TextIOWrapper name='./MIB_Files/BaseAgents.csv' mode='r' encoding='UTF-8'>,
 'BaseAliens': <_io.TextIOWrapper name='./MIB_Files/BaseAliens.csv' mode='r' encoding='UTF-8'>,
 'BaseCabines': <_io.TextIOWrapper name='./MIB_Files/BaseCabines.csv' mode='r' encoding='UTF-8'>,
 'BaseGardiens': <_io.TextIOWrapper name='./MIB_Files/BaseGardiens.csv' mode='r' encoding='UTF-8'>,
 'BaseMiams': <_io.TextIOWrapper name='./MIB_Files/BaseMiams.csv' mode='r' encoding='UTF-8'>,
 'BaseResponsables': <_io.TextIOWrapper name='./MIB_Files/BaseResponsables.csv' mode='r' encoding='UTF-8'>}
\end{pythoncode}


\end{frame}


\begin{frame}[fragile]
\begin{pythoncode}
mon_chemin = input('Quel est le chemin relatif du répertoire contenant les fichiers csv ?\n')
#"./MIB_Files/"

mon_alias = input('Alias du fichier py créé (sera ./MaBase_alias.py) ?\n')

mon_fic = "MaBase_%s.py" % mon_alias
 
mes_csv_file = {Path(f).stem:open(f,"r") for f in glob.glob(mon_chemin + "*.csv")}

mes_csv = {Path(f).stem:open(f,"r").readlines() for f in glob.glob(mon_chemin + "*.csv")}

mon_py = open(mon_fic,"w+")
\end{pythoncode}
\end{frame}




\begin{frame}[fragile]
\begin{pythoncode}
In [60]: mes_csv
Out[60]: 
{'BaseAgents': ['Nom,Ville\n',
  'Branno,Terminus\n',
  'Darell,Terminus\n',
  'Demerzel,Uco\n',
  'Seldon,Terminus\n',
  'Dornick,Kalgan\n',
  'Hardin,Terminus\n',
  'Trevize,Hesperos\n',
  'Pelorat,Kalgan\n',
  'Riose,Terminus\n'],
 'BaseAliens': ['Nom,Sexe,Planete,NoCabine\n',
  'Zorglub,M,Trantor,1\n',
  'Blorx,M,Euterpe,2\n',
  'Urxiz,M,Aurora,3\n',
  'Zbleurdite,F,Trantor,4\n',
  'Darneurane,M,Trantor,4\n',
  'Mulzo,M,Helicon,6\n',
  'Zzzzzz,F,Aurora,7\n',
  'Arghh,M,Nexon,8\n',
  'Joranum,F,Euterpe,9\n'],
 'BaseCabines': ['NoCabine,NoAllee\n',
  '1,1\n',
  '2,1\n',
  '3,1\n',
  '4,1\n',
  '5,1\n',
  '6,2\n',
  '7,2\n',
  '8,2\n',
  '9,2\n'],
 'BaseGardiens': ['Nom,NoCabine\n',
  'Branno,1\n',
  'Darell,2\n',
  'Demerzel,3\n',
  'Seldon,4\n',
  'Dornick,5\n',
  'Hardin,6\n',
  'Trevize,7\n',
  'Pelorat,8\n',
  'Riose,9\n'],
 'BaseMiams': ['NomAlien,Aliment\n',
  'Zorglub,Bortsch\n',
  'Blorx,Bortsch\n',
  'Urxiz,Zoumise\n',
  'Zbleurdite,Bortsch\n',
  'Darneurane,Schwanstucke\n',
  'Mulzo,Kashpir\n',
  'Zzzzzz,Kashpir\n',
  'Arghh,Zoumise\n',
  'Joranum,Bortsch\n'],
 'BaseResponsables': ['NoAllee,Nom\n', '1,Seldon\n', '2,Pelorat\n']}
\end{pythoncode}


\end{frame}


\begin{frame}[fragile]
\begin{pythoncode}
mon_py = open(mon_fic, "w+")
\end{pythoncode}


  

\end{frame}






\begin{frame}[fragile]
\begin{pythoncode}
In [63]: b = 'BaseGardiens'

In [64]: mes_csv[b]
Out[64]: 
['Nom,NoCabine\n',
 'Branno,1\n',
 'Darell,2\n',
 'Demerzel,3\n',
 'Seldon,4\n',
 'Dornick,5\n',
 'Hardin,6\n',
 'Trevize,7\n',
 'Pelorat,8\n',
 'Riose,9\n']
\end{pythoncode}


\end{frame}



\begin{frame}[fragile]
\begin{pythoncode}
In [65]: lignes = mes_csv[b]

In [66]: lignes[2]
Out[66]: 'Darell,2\n'
\end{pythoncode}
  \pause

  \begin{pythoncode}
In [67]: lignes[2].split()
Out[67]: ['Darell,2']
\end{pythoncode}

  \pause

  \begin{pythoncode}
In [68]: lignes[2].split()[0]
Out[68]: 'Darell,2'
\end{pythoncode}

  \pause

  \begin{pythoncode}
In [69]: lignes[2].split()[0].split(',')
Out[69]: ['Darell', '2']
\end{pythoncode}



\end{frame}






\begin{frame}[fragile]
\begin{pythoncode}
def creer_classes():
    for b in mes_csv:
        mon_py.write("class " + b[4:-1] + ":\n\tdef __init__(self")
        lignes = mes_csv[b]
        attributs = lignes[0].split()[0].split(',')
        for a in attributs:
            mon_py.write(", " + a)
        mon_py.write("):\n\t\t")
        for a in attributs:
            mon_py.write("self.%s = %s\n\t\t" % (a,a))
        mon_py.write("\n\n")
\end{pythoncode}


\end{frame}




\begin{frame}[fragile]
\begin{pythoncode}
class Alien:
	def __init__(self, Nom, Sexe, Planete, NoCabine):
		self.Nom = Nom
		self.Sexe = Sexe
		self.Planete = Planete
		self.NoCabine = NoCabine


class Cabine:
	def __init__(self, NoCabine, NoAllee):
		self.NoCabine = NoCabine
		self.NoAllee = NoAllee
\end{pythoncode}


\end{frame}


\begin{frame}[fragile]
\begin{pythoncode}
def creer_bases():
    for b in mes_csv:
        nom = b[4:-1]
        mon_py.write(b + " = { ")
        lignes = mes_csv[b]
        for index,ligne in enumerate(lignes[1:]):
            ligne = ligne.split()[0].split(',')
            debut = '' if index == 0 else ', '
            mon_py.write(debut + nom + "(")
            for att in ligne[:-1]:
                mon_py.write("'" + att + "', ")
            mon_py.write("'" + ligne[-1] +"')")
        mon_py.write(" }\n\n")
\end{pythoncode}


\end{frame}




\begin{frame}[fragile]
\begin{pythoncode}
BaseCabines = { Cabine('1', '1'), Cabine('2', '1'), Cabine('3', '1'), Cabine('4', '1'), Cabine('5', '1'), Cabine('6', '2'), Cabine('7', '2'), Cabine('8', '2'), Cabine('9', '2') }

BaseMiams = { Miam('Zorglub', 'Bortsch'), Miam('Blorx', 'Bortsch'), Miam('Urxiz', 'Zoumise'), Miam('Zbleurdite', 'Bortsch'), Miam('Darneurane', 'Schwanstucke'), Miam('Mulzo', 'Kashpir'), Miam('Zzzzzz', 'Kashpir'), Miam('Arghh', 'Zoumise'), Miam('Joranum', 'Bortsch') }
\end{pythoncode}


\end{frame}



\begin{frame}[fragile]
\begin{pythoncode}
def ferme():
    for b in mes_csv_file:
        mes_csv_file[b].close()
    mon_py.close()
\end{pythoncode}
\end{frame}

\begin{frame}[fragile]
\begin{pythoncode}
In [70]: villes = { agent.Ville for agent in BaseAgents }

In [71]: villes
Out[71]: {'Hesperos', 'Kalgan', 'Terminus', 'Uco'}
\end{pythoncode}


\end{frame}

\begin{frame}
  À vous de jouer...
\end{frame}


\begin{frame}[fragile]
\begin{pythoncode}
# -1 l'ensemble des gardiens ;
gardiens = { gardien.nom for gardien in baseGardien }
\end{pythoncode}
\end{frame}

\begin{frame}[fragile]
\begin{pythoncode}
# -2 l'ensemble des villes où habitent les agents ;
villes = { agent.ville for agent in baseAgent }
\end{pythoncode}
\end{frame}

\begin{frame}[fragile]
\begin{pythoncode}
# -3 l'ensemble des triplets (no de cabine,alien,gardien) pour chaque cabine ;
triples = { (alien.no_cabine, alien.nom, gardien.nom)
            for alien in baseAlien
            for gardien in baseGardien if gardien.no_cabine == alien.no_cabine}
\end{pythoncode}
\end{frame}

\begin{frame}[fragile]
\begin{pythoncode}
# -4 l'ensemble des couples (alien,allée) pour chaque alien ;
# -5 l'ensemble de tous les aliens de l'allée 2 ;
# -6 l'ensemble de toutes les planètes dont sont originaires les aliens habitant une cellule de numéro pair ;
# -7 l'ensemble des aliens dont les gardiens sont originaires de Terminus ;
# -8 l'ensemble des gardiens des aliens féminins qui mangent du bortsch ;
# -9 l'ensemble des cabines dont les gardiens sont originaires de Terminus ou dont les aliens sont des filles
 \end{pythoncode}
\end{frame}

\begin{frame}[fragile]
\begin{pythoncode}
 
# Y a-t-il 
# -10 des aliments qui commencent par la même lettre que le nom du gardien qui surveille l'alien qui les mange ?
test10 = True in { miam.aliment[0] == gardien.nom[0]
                   for miam in baseMiam
                   for alien in baseAlien
                   for gardien in baseGardien
                   if  gardien.no_cabine  ==  alien.no_cabine and  alien.nom  ==
                   miam.nom_alien }
\end{pythoncode}
\end{frame}

\begin{frame}[fragile]
\begin{pythoncode}
# -11 des aliens originaires d'Euterpe ?
 
# Est-ce que 
# -12 tous les aliens ont un 'x' dans leur nom ?
\end{pythoncode}
\end{frame}

\begin{frame}[fragile]
\begin{pythoncode}
# -13 tous les aliens qui ont un 'x' dans leur nom ont un gardien qui vient de Terminus ?
test13 = False not in { agent.ville == 'Terminus' and 'x' in alien.nom
                        for alien in baseAlien
                        for gardien in baseGardien
                        for agent in baseAgent
                        if gardien.no_cabine == alien.no_cabine and gardien.nom == agent.nom }
# -14 il existe un alien masculin originaire de Trantor qui mange du Bortsch ou dont le gardien vient de Terminus
\end{pythoncode}
\end{frame}










\end{document}
%%% Local Variables:
%%% mode: latex
%%% TeX-master: t
%%% End:
